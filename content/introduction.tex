% ! TEX root = ./main.tex 

\section{更新}
\begin{itemize}
  \item 支持中文标题自定义分行
  \item 增加了更现代化的键值对接口用于添加信息和其他模板的配置
  \item 删除了一些原有的接口,比如图表目录的生成配置
\end{itemize}
\section {TODO}
\begin{itemize}
  \item 支持封面自定义,在keys中添加键值设置
  \item 支持多字体设置,{\ttfamily fontset}键值好像不会生效
  \item 未来将不继承{\ttfamily ctexart}模板,{\ttfamily ctexset}的接口将被删除
  \item 支持数学字体自定义,添加数学字体设置的键值{\ttfamily math}
\end{itemize}

\section{写在前面}
\subsection{使用方法}
你只需要修改模板文件夹下的{\ttfamily main.tex}内容即可,其他不需要修改。
{\color{red} 注意,在清理文件时,请勿清理{\ttfamily NJUSTThesis.cls}文件,此为模板文件。}
\subsection{编译方式}
参考文献后端采用 {\ttfamily biber},模板继承至{\ttfamily ctexart},
所以编译方式为

{\ttfamily xelatex <jobname.tex> -> biber <jobname> xelatex <jobname.tex> 
-> xelatex <jobname.tex>}

或

{\ttfamily latexmk -xelatex <jobname.tex>}

\subsection{字体}
根据南京理工大学学士论文撰写格式,中文采用宋体({\ttfamily SimSun}),
英文采用Times New Roman({Times New Roman}),
所以请确保你安装了上述两种字体。你可以根据下列命令查看是否安装了上述两种字体
\begin{lstlisting}[language=bash]
    fc-list :lang=zh # 查看中文字体
    fc-list :lang=en # 查看英文字体
    # 或者将输出内容写入txt文件
    fc-list > font.txt 
\end{lstlisting}

如果选择模板生成封面,你需要安装{\ttfamily 方正魏碑简体}字体,该字体为WPS中的可用字体,
已提供在{\ttfamily font}文件夹中。此外{\ttfamily font}还提供了一些其他字体,比如
\begin{enumerate}
    \item {\ttfamily XITS} 类罗马风格的数学字体
    \item {\ttfamily math time pro 2} 类罗马风格的数学字体
\end{enumerate}
安装字体请右键为所有用户安装。

{\color{red} \heiti 注意:{\ttfamily mtpro2}宏包会和{\ttfamily unicode-math}宏包
冲突,请选择好数学字体}。

此外模板提供了一些字体命令
\begin{enumerate}
    \item \lstinline|\kaiti|,{\kaiti 楷体}
    \item \lstinline|\heiti|,{\heiti 黑体}
    \item \lstinline|\weibei|,{\weibei 方正魏碑简体}
\end{enumerate}

\section{NJUSTThesis模板介绍}
\subsection{一些基本知识}
下面是{\ttfamily main.tex}文件中的文件格式的
介绍,你需要了解这些基本知识。
\begin{lstlisting}
  \documentclass{NJUSTThesis}
  % 这里是导言区
  % 一些预定义格式和宏包引用放在该处

  \begin{document}
  % 这里是正文内容,你所写的论文就放在该处
  \end{document}
\end{lstlisting}
\subsection{标题}
在开始之前必须要在导言区指定指定论文标题,包括中文和英文,
\begin{lstlisting}
    \title{南京理工大学本科学士学位论文\LaTeX{}模板}
    % 中文标题通过\title命令指定
    \englishtitle{Nanjing University of Science \& Technology \\ \LaTeX{} Template}
    % 英文标题通过\englishtitle命令指定
\end{lstlisting}
需要注意的是,中文标题不接受任何的换行符号,
中文标题会自动换行,每19个字符换行;英文标题接受换行符号,默认不换行。

\subsection{封面}
模板提供了两种加入封面的方式,一种是模板自己生成的封面,另外一种是
由作者自己嵌入封面。
\begin{lstlisting}
    \NJUSTCover % 模板生成封面的命令
    % 自定义插入封面
    \usepackage{pdfpages}% 必须使用该宏包
    \includepdf[pagecommand=\thispagestyle{empty}]{<filename>}
\end{lstlisting}

\subsubsection{封面信息}
虽然在没有指定信息时依然可以生成封面,但是最好指定作者、导师等信息
,以免出现不必要的编译错误。{\ttfamily V0.2.0}版本的模板提供了更
现代化的键值来设置信息,如下
\begin{lstlisting}
  \njustsetup{
    info = {
      title = {
        zh = {<中文标题>},
        en = {<英文标题>},
      },
      author = {
        zh = {<中文名字>},
        en = {<英文名字>},
      },
      tutor = {
        school/zh = {<校内指导老师>},
        school/en = {<校内指导老师>},
        school/level = {<校内指导老师职称>},
        external/zh = {<校外指导老师>},
        external/en = {<校外指导老师>},
        external/level = {<校外指导老师职称>},
        },
        number = {<学号>},
        date = {<时间>},%可以留空,date = {}
        major = {<专业>},
        school = {<学院>},
        direction = {<研究方向>},
    },
    figtab = <true|false>, % 用于设置是否生成图表目录
    keywords/zh = {<中文关键词>},
    keywords/en = {<英文关键词>},
  }
\end{lstlisting}

此外,也保留了原来的接口设置信息,即
\begin{lstlisting}
    \authorname{<中文名字>}{<English name>}
    \tutor{<校内导师>}{<校外导师>}
    \post{<校内导师职称>}{<校外导师职称>}
    \englishtutor{<校内导师>}{<校外导师>}
    \major{<学生专业>}
    \direction{<研究方向>}
    \school{<学生学院>}
    % 可以不指定论文提交时间,则时间为编译时的时间,
    % 也可指定时间,但是只接受2025年6月格式
    % 不接受具体到日
    \date{<论文提交时间>}
\end{lstlisting}

\subsection{摘要和关键词}
模板提供了英文和中文的摘要环境,有一个可选项,
{\ttfamily <chinese>} 和 {\ttfamily <english>},
默认均为{\ttfamily chinese}。
\begin{lstlisting}
    \begin{abstract}[chinese]
        % 中文摘要
    \end{abstract}
    \begin{abstract}[english]
        % 英文摘要
    \end{abstract}
\end{lstlisting}


\subsection{目录}
模板定制了目录样式,包括正文目录和图表目录。
\begin{lstlisting}
    \njustsetup{
    figtab = <true|false>,
    }
    \NJUSTContents
\end{lstlisting}
该命令没有任何参数,
{\ttfamily figtab}接受布尔值,
{\ttfamily <true>} 和 {\ttfamily <false>}
默认值为{\ttfamily <true>},代表是否生成
图表目录。

\section{正文}
模板提供了四级标题,其中第四级标题不会加入目录。
\begin{lstlisting}
    \section{<一级标题>}
    \subsection{<二级标题>}
    \subsubsection{<三级标题>}
    \subsubsubsection{<四级标题>}
\end{lstlisting}
建议最好不要使用四级标题(条)。
其余按照\LaTeX{}格式书写即可。

\subsection{字体}
中文字体为宋体,英文字体为Times New Roman。
其余字体未指定,需要自行指定。
\begin{lstlisting}
    \setCJKmonofont{<中文等宽字体>}
\end{lstlisting}

提供了一些字体命令。

\subsection{数学公式}
模板未指定数学字体,默认字体为{\ttfamily Computer Modern / Latin Modern}。
如果需要修改数学字体请按照如下方法修改:
\begin{lstlisting}
    % 使用Times New Roman
    \usepackage{unicode-math}
    \setmathfont{Times New Roman}

    % 使用类新罗马字体
    \usepackage{unicode-math}
    \setmathfont{XITS Math} % 需要安装 XITS Math 字体
    
    % 使用math time pro2 字体
    \usepackage{mtpro2} % 需要安装该宏包
\end{lstlisting}

在输入数学公式前,需要加载{\ttfamily amsmath}宏包,
该宏包提供了很多数学相关的环境和命令。
\begin{lstlisting}
    % 该宏包已在模板中加载,可以不加载
    \usepackage{amsmath}
\end{lstlisting}

\subsubsection{编号公式}
\Cref{eq:continuity Equation} 是连续方程。
\begin{equation}
    \frac{\partial}{\partial t}\oiiint\limits_{\mathcal{V}} \rho 
    \,\mathrm{d} \mathcal{V} +\oiint\limits _S \rho \mathbf{V}\cdot 
    \mathbf{dS} =0
    \label{eq:continuity Equation}
\end{equation}

\begin{lstlisting}
    \begin{equation}
        \frac{\partial}{\partial t}\oiiint\limits_{\mathcal{V}} \rho 
        \,\mathrm{d} \mathcal{V} +\oiint\limits _S \rho \mathbf{V}\cdot 
        \mathbf{dS} =0
        \label{eq:continuity Equation}
    \end{equation}
\end{lstlisting}

\subsubsection{无编号公式}
\subsubsubsection{行内公式和行间公式}
行内公式效果$\frac{1}{2}$,$\rho$,行内公式默认比较紧凑,可以开启展示模式
$\displaystyle  \frac{1}{2}$,行间公式 
\[
    \begin{bmatrix}
        \alpha & \beta \\ 
        \rho   & \gamma 
    \end{bmatrix}
\]

\begin{lstlisting}
    $\frac{1}{2}$
    $\rho$
    $\displaystyle  \frac{1}{2}$
    \[
        \begin{bmatrix}
            \alpha & \beta \\ 
            \rho   & \gamma 
        \end{bmatrix}
    \]
\end{lstlisting}

\subsubsubsection{全局展示模式}
如果想要开启全局展示模式,需要在导言区增加如下命令
\begin{lstlisting}
    \everymath{\displaystyle}
\end{lstlisting}

\subsection{图片}
建议插入{\ttfamily pdf}格式的矢量图片,\LaTeX{}接受
{\ttfamily pdf},{\ttfamily png},{\ttfamily jpg},
{\ttfamily eps}等图片格式,不接受{\ttfamily svg}格式的
矢量图片。

如下\Cref{fig:B787}是波音公司的梦想客机(dreamliner)。
\begin{figure}[!ht]
    \centering 
    \includegraphics[width=\textwidth]{./figure/787-dreamliner.jpeg}    
    \caption{波音B787,Dreamliner梦想客机}
    \label{fig:B787}
\end{figure}

\begin{lstlisting}
    \begin{figure}[!ht]
        \centering 
        \includegraphics[width=\textwidth]{./figure/787-dreamliner.jpeg}    
        \caption{波音B787,Dreamliner梦想客机} % \caption 必须在 \label之前,否则不能正常交叉引用
        \label{fig:B787}
    \end{figure}   
\end{lstlisting}

也可以插入一些子图,如下\Cref{fig:A350} 
\begin{figure}[!ht]
    \centering 
    \begin{subcaptionblock}{0.48\textwidth}
        \centering 
        \includegraphics[width=\linewidth]{./figure/A359.jpg}
        \caption{Airbus A350-900}
        \label{fig:A359}
    \end{subcaptionblock}
    \begin{subcaptionblock}{0.48\textwidth}
        \centering
        \includegraphics[width=\linewidth]{./figure/A35k.jpg}
        \caption{Airbus A350-1000}
        \label{fig:A35k}
    \end{subcaptionblock}
    \caption{Airbus A350家族}
    \label{fig:A350}
\end{figure}

\begin{lstlisting}
\begin{figure}[!ht]
    \centering 
    \begin{subcaptionblock}{0.48\textwidth}
        \centering 
        \includegraphics[width=\linewidth]{./figure/A359.jpg}
        \caption{Airbus A350-900}
        \label{fig:A359}
    \end{subcaptionblock}
    \begin{subcaptionblock}{0.48\textwidth}
        \centering
        \includegraphics[width=\linewidth]{./figure/A35k.jpg}
        \caption{Airbus A350-1000}
        \label{fig:A35k}
    \end{subcaptionblock}
    \caption{Airbus A350家族}
    \label{fig:A350}
\end{figure} 
\end{lstlisting}

\subsection{表格}
论文一般要求插入三线表,需要在导言区引用宏包{\ttfamily booktabs}
(该宏包已在V0.2.1版本之后,由模板添加,所以不需要在导言区引用)
\begin{lstlisting}
    \usepackage{booktabs}
\end{lstlisting}

比如下\Cref{tab:example}
\begin{table}[!ht]
    \centering 
    \caption{示例表}
    \label{tab:example}
    \setlength{\tabcolsep}{5em}{
    \begin{tabular}{cc}
        \toprule
        项目 & 值\\
        \midrule
        质量(Kg) & 2 \\
        长度(m) & 2 \\ 
        \bottomrule
    \end{tabular}}
\end{table}

\begin{lstlisting}
\begin{table}[!ht]
    \centering 
    \caption{示例表}
    \label{tab:example}
    \setlength{\tabcolsep}{5em}{
    \begin{tabular}{cc}
        \toprule
        项目 & 值\\
        \midrule
        质量(Kg) & 2 \\
        长度(m) & 2 \\ 
        \bottomrule
    \end{tabular}}
\end{table}   
\end{lstlisting}

\thanks
模板提供了致谢环境。
\begin{lstlisting}
    \thanks
\end{lstlisting}

\newpage
\subsection{参考文献}
模板使用参考文献的样式为GB7714-2015,符合论文要求。
通过 {\ttfamily BibLaTeX}管理参考文献,需要指定
{\ttfamily bib}文件\cite{aerospace11060494}
,通过\lstinline|\cite{<name>}|
引用参考文献\cite{FHLX200903002}。
\begin{lstlisting}
    % 指定bib文件,需要放在导言区
    \addbibresource[location=local]{reference.bib}
    
    % 打印参考文献
    % 这句命令必须要添加在 \end{document} 命令之前
    % 以打印参考文献条目,否则将不会打印参考文献
    % 该命令你可以放在任何地方,则会在该命令的当前
    % 打印参考文献的条目
    % NOTE:该命令不可使用两次,以防出现不可预知的
    % 错误,即该命令只能在全文中出现一次
    % 为加快编译速度,建议仅用xelatex编译,最后定稿
    % 时,再使用编译链编译,以生成正确的交叉引用
    \printreference
\end{lstlisting}

注意,要正确编译,产生参考文献,必须要采用如下编译方式:

{\ttfamily xelatex <jobname.tex> -> biber <jobname> -> xelatex 
<jobname.tex> -> xelatex <jobname.tex>}

{\ttfamily bib}后端为 {\ttfamily biber}。
\printreference

\newpage
\appendix

\section{其他设置}
\subsection{附录}
如果你需要添加附录,请在导言区引用{\ttfamily appendix}宏包
\begin{lstlisting}
    \usepackage{appendix}

    % 在需要添加附录的部分添加如下命令
    \newpage % 另起一页
    \appendix 

    % 后面正常按照正文书写即可
    \section{}
\end{lstlisting}

\subsection{代码抄录}
模板没有定制代码抄录环境,需要自行定制样式
\begin{lstlisting}
    % 使用该宏包定制样式
    \usepackage{listings}
\end{lstlisting}

\subsection {节标题分页}
模板默认节标题与上一节内容是相连的,如果想要新的一节另起一页,有如下两个
方法,
\begin{enumerate}
    \item 在新的节标题前面添加命令 \lstinline|\newpage|
    \item 在导言区添加命令 \lstinline|\ctexset{section/break=\clearpage}|
\end{enumerate}
二者任选其一即可。
